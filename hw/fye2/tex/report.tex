\documentclass{../../tex_template/asaproc}
\usepackage{graphicx} % \includegraphics
\usepackage{float}    % To keep figures in right place. 
                      % Usage: \being{figure}[H] \includegraphics{tmp.pdf} \end{figure}
\usepackage{subfig}   % \subfloat
\usepackage{amsmath}  % bmatrix, pmatrix, etc
\usepackage{bm}
\newcommand{\p}[1]{\left(#1\right)}
\newcommand{\bk}[1]{\left[#1\right]}
\newcommand{\bc}[1]{ \left\{#1\right\} }
\newcommand{\abs}[1]{ \left|#1\right| }
\newcommand{\norm}[1]{ \left|\left|#1\right|\right| }
\newcommand{\E}{ \text{E} }
\newcommand{\N}{ \mathcal N }
\newcommand{\ds}{ \displaystyle }

%\usepackage{times}
%If you have times installed on your system, please
%uncomment the line above

%For figures and tables to stretch across two columns
%use \begin{figure*} \end{figure*} and
%\begin{table*}\end{table*}
% please place figures & tables as close as possible
% to text references

\newcommand{\be}{\begin{equation}}
\newcommand{\ee}{\end{equation}}
\newcommand{\y}{\bm y}
\newcommand{\M}{\mathcal{M}}
\usepackage{verbatim}

\title{FYE--- Ozone}

%input all authors' names
\author{
  Test ID: 911$^1$\\
  University California -- Santa Cruz$^1$\\
}

%input affiliations
%{USDA Forest Service Forest Products Laboratory}

\begin{document}
\maketitle
\begin{abstract}
This should contain a summary of all my work

\begin{keywords}
Add some key words,
Bayesian Regression, ozone, temperature, wind speed, radiation, g-priors
\end{keywords}
\end{abstract}

\section{Introduction}
An intro...

\section{Exploratory Analysis}
Figure \ref{fig:tcy} shows, for each of the six studies, the mortality rates of
the control groups (red) and treatment (blue) groups along with the difference
in mortality rates $y_i = C_i-T_i$ (green). \textbf{Across five of the six studies
(i.e. most of the studies), the mortality rates of the control group are higher
than that of the treatment group.} (Notice how the red line is usually higher
than the blue line, and the green line is usually above 0.) The only exception
is the AMIS study group -- the control group has a mortality rate 9.7, which is
slightly less than that of the treatment group (10.85). It certainly \textbf{seems
that aspirin had an effect on most of the RCTs}. More investigation will be
required to understand why the mortality rates of the control group were lower
than that of the treatment group in the AMIS study. Especially since the AMIS
study had the most participant -- at least double that of any other study.

\begin{figure}[H]
  \includegraphics[scale=.5]{img/tcy.pdf}
  \caption{\small Mortality rates for control (red) and treatment (blue) groups
  for different randomized control trials. The green line shows the difference
  in mortality rates between the control group and the treatment group (control -
  treatment). A lower mortality rate is better.}
  \label{fig:tcy}
\end{figure}

Figure \ref{fig:m1Post} shows the posterior distribution of the mean parameter
$\mu$ in $\M_1$. $p=P\bk{\mu>0|\bm{y},\M_1}$ = .9546. A large
value for $p$ favors the aspirin treatment over the control. So, we would
conclude that mortality following a heart attack can be reduced by taking a low
does of aspirin daily.

\begin{figure}[H]
  \includegraphics[scale=.5]{img/m1Post.pdf}
  \caption{\small Posterior distribution for $\mu$ in $\M_1$.
  Posterior mean = .0098. Posterior standard deviation = .0058.
  95\% HPD = (-.00156,.0212). $P\bk{\mu>0|\bm{y},\M_1}$ = .9546.}
  \label{fig:m1Post}
\end{figure}



\section{Methods}
Some methods...

\subsection{Models}

$$
\begin{array}{lrcl}
  \M_1: \\
  & y_i | \mu &\sim& N(\mu_i,V_i)\\
  & p(\mu) &\propto& 1\\
  \\
  \M_2: \\
  & y_i | \theta_i &\sim& N(\theta_i,V_i)\\
  & \theta_i | \mu &\sim& N(\mu,\sigma^2)\\
  & p(\mu) &\propto& 1\\
  & \sigma^2 &\sim& IG(3,1)\\
\end{array}
$$

$\M_2$ \textbf{is a special case of} $\M_1$. Note that $\M_2$ is
obtained when 

\begin{figure}[H]
  \includegraphics[scale=.5]{img/m2MuS2Post.pdf}
  \caption{say...}
  \label{fig:m2MuS2Post}
\end{figure}

\begin{figure}[H]
  \includegraphics[scale=.5]{img/thetaPost.pdf}
  \caption{say...}
  \label{fig:thetaPost}
\end{figure}


\section{Analysis}
Some analysis

\section{Conclusions}
Some conclusion...

\begin{references}
{\footnotesize
\itemsep=3pt
\item {\em Zellner, Arnold. On assessing prior distributions and Bayesian regression analysis with g-prior distributions. Bayesian inference and decision techniques: Essays in Honor of Bruno De Finetti 6 (1986): 233-243.}
\item {\em Gelman, A., Carlin, J. B., Stern, H. S., \& Rubin, D. B. (2014). Bayesian data analysis (Vol. 2). Boca Raton, FL, USA: Chapman \& Hall/CRC, 73.}
}
\end{references}

\newpage
\section{Source Code for Gibbs Sampler used in $\M_2$}
\verbatiminput{../src/gibbs.R}


\end{document}

%\begin{figure*}
%  \centering
%  \includegraphics[scale=.55]{figs/mapDat.pdf}
%  \vspace{-7em}
%  \caption{\small Some Caption.}
%  \label{fig:mapDat}
%\end{figure*}

%\begin{figure}[H]
%  \includegraphics[scale=.5]{figs/pairsLogRate.pdf}
%  \caption{\small Hi Motor vehicle theft is not strongly correlated with any other thefts.}
%  \label{fig:logOdds}
%\end{figure}

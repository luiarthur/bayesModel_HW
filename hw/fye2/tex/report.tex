\documentclass{../../tex_template/asaproc}
\usepackage{graphicx} % \includegraphics
\usepackage{float}    % To keep figures in right place. 
                      % Usage: \being{figure}[H] \includegraphics{tmp.pdf} \end{figure}
\usepackage{subfig}   % \subfloat
\usepackage{amsmath}  % bmatrix, pmatrix, etc
\usepackage{amsfonts} % \mathbb{Q}, etc.
\usepackage{bm}
\newcommand{\p}[1]{\left(#1\right)}
\newcommand{\bk}[1]{\left[#1\right]}
\newcommand{\bc}[1]{ \left\{#1\right\} }
\newcommand{\abs}[1]{ \left|#1\right| }
\newcommand{\norm}[1]{ \left|\left|#1\right|\right| }
\newcommand{\E}{ \text{E} }
\newcommand{\N}{ \mathcal N }
\newcommand{\ds}{ \displaystyle }

%\usepackage{times}
%If you have times installed on your system, please
%uncomment the line above

%For figures and tables to stretch across two columns
%use \begin{figure*} \end{figure*} and
%\begin{table*}\end{table*}
% please place figures & tables as close as possible
% to text references

\newcommand{\be}{\begin{equation}}
\newcommand{\ee}{\end{equation}}
\newcommand{\y}{\bm y}
\newcommand{\M}{\mathcal{M}}
\usepackage{verbatim}
\newcommand{\Q}{\mathbb{Q}}
\newcommand{\simi}{\overset{ind.}{\sim}}
\newcommand{\iid}{\overset{iid}{\sim}}

\title{FYE--- Ozone}

%input all authors' names
\author{
  Test ID: 911$^1$\\
  University California -- Santa Cruz$^1$\\
}

%input affiliations
%{USDA Forest Service Forest Products Laboratory}

\begin{document}
\maketitle
\begin{abstract}
$\Q$: \textbf{Can mortality following a heart attach be reduced by taking a low
dose of aspirin daily?} Using the data from six randomized controlled trials
(RCTs), and two models ($\M_1$ -- fixed effects model and $\M_2$ --
random-effects model) the question ($\Q$) was studied. The following is a
summary of the findings. (a) Visually, it seems that aspirin had a consistent
effect in most of the RCTs (see Figure \ref{fig:tcy}). But more investigation
is needed to determine whether the effect is large in clinical terms. (b) The
probability that the control groups' mortality rate is higher than that of the
treatment groups across all six studies computed using a fixed effects model
and given the provided data = $P\bk{\mu > 0 | y, \M_1} = 95.46\%$. This
suggests that under $\M_1$, the mortality for patients (in the six RCTs) who
take low doses of aspirin following a heart attack is reduced in most cases
(95\% of the time). (c)(i) $\M_1$ is a special case of $\M_2$. Specifically,
when $\mu$ has the same prior in both models and $\sigma^2 \rightarrow 0$ in
$\M_2$, $\M_2$ becomes $\M_1$. (i.e. if the $\theta_i$'s are all the same in
$\M_2$, then $\M_1$ is retrieved.) A simple way to decide if $\M_2$ is better
than $\M_1$ is to compare the DIC's of the two models. The model with the
smaller DIC is preferred. (c)(ii) The probability that the control groups'
mortality rate is higher than that of the treatment groups across all six
studies computed using a random effects model and given the provided data =
$P\bk{\mu > 0 | y, \M_2} = 57.49\%$. This suggests that under $\M_2$, the
mortality for patients (in the six RCTs) who take low doses of aspirin
following a heart attack is reduced in 57.49\% of the cases. (The full
conditionals for each parameter in $\M_2$ are included in this paper.)

\begin{keywords}
Random effects vs fixed effects models, aspirin, heart attack.
\end{keywords}
\end{abstract}

\section{Introduction}
People with heart problems commonly use aspirin to reduce blood pressure and
the chances of a heart attack. In the early 1990's when aspirin was known to
have a blood-thinning effect. But it was not known if the drug saved lives.  A
dataset containing the data of 6 randomized controlled trials (RCTs) were used
to analyze the effectiveness of aspirin in reducing mortality after a heart
attack. The 6 studies included were UK-1, CDPA, GAMS, UK-2, PARIS, and AMIS.
The data included the number patients in the studies (both control and
treatment groups), as well as the all-cuase mortality rates for both groups.
The control and treatment groups were divided farily evenly. But the number
of patients were not all equal. Particularly, the number of patients in the AMIS 
is more than double that of any other study. Care should be taken to make
use of such information. \textbf{Write order of material presented}\\

\section{Exploratory Analysis}
To better understand the data, I first visualized it.  Figure \ref{fig:tcy}
shows, for each of the six studies, the mortality rates of the control groups
(red) and treatment (blue) groups along with the difference in mortality rates
$y_i = C_i-T_i$ (green).  \textbf{Across five of the six studies (i.e. most of
the studies), the mortality rates of the control group are higher than that of
the treatment group.} (Notice how the red line is usually higher than the blue
line, and the green line is usually above 0.) The only exception is the AMIS
study group -- the control group has a mortality rate 9.7, which is slightly
less than that of the treatment group (10.85).  It certainly \textbf{seems that
aspirin had an effect on most of the RCTs}. More investigation will be required
to understand why the mortality rates of the control group were lower than that
of the treatment group in the AMIS study. Especially since the AMIS study had
the most participant -- at least double that of any other study. I would 
say the \textbf{effect of aspirin appears to be great in clinical terms}
because for the studies where aspirin seemed to have a positive effect, the
mortality rates of the treatment group were 2.4\% lower than that of the
control group on average. While 2.4\% may not seem like a large number, 
the life of every person is of great value.\\

\begin{figure}[H]
  \includegraphics[scale=.5]{img/tcy.pdf}
  \caption{\small Mortality rates for control (red) and treatment (blue) groups
  for different randomized control trials. The green line shows the difference
  in mortality rates between the control group and the treatment group (control -
  treatment). A lower mortality rate is better.}
  \label{fig:tcy}
\end{figure}

It should be expected that different study groups will have different mortality
rates, due to different demographics. But the fact that the mortality rates are
lower than that of the treatment group by a more-or-less constant amount may be
indicating that aspirin has a constant or fixed effect on mortality rates.
More formal statistical analysis will be needed to make more informative
statements.\\

This concludes the exploratory analysis. In the next section, I describe the
statisical models used to study the data further.\\

\section{Models} 
To answer the question ($\Q$) of interest, two models were fit to the data --
(1) a fixed-effects model and (2) a random-effects model. I will describe each
of them in the two following subsections.\\

\subsection{Fixed-effects Model}
First, a fixed-effects model was used to combine the information from the six
(similar but different) studies. This model assumes no between-experiment
heterogeneity. (i.e. effects are assumed to be constant across trials.) From
Figure \ref{fig:tcy}, we can reason that homogeneity between experiments is not
a good assumption. We will address the difference in study groups with
random-effects models in the next section. I have named the fixed-effects
model $\M_1$ and given it the following form:

$$
\begin{array}{lrcl}
  \M_1: \\
  & y_i | \mu &\simi& N(\mu,V_i),~\text{for$~i = 1,...6$}\\
  & p(\mu) &\propto& 1\\
\end{array}
$$

where $y_i = (C_i - T_i)$, $C_i$ is the mortality rate in the control group for
study $i$, and $T_i$ is the mortality rate in the treatment group for study
$i$. Note that if $y_i > 0$ then the mortality rate of study $i$ is higher
for the control group, suggesting that the treatment is reducing the
mortality rates of subjects in the study.\\

In this model, the sampling distribution for each difference ($y_i$) is assumed
to be normal with mean $\mu$ (shared across all RCTs -- hence fixed-effects)
and individual (binomial) variances $V_i$. The variances are treated as known
since the sample sizes in each RCT are large enough. Specifically,

$$
V_i = \frac{C_i(1-C_i)}{n_{C_i}} + \frac{T_i(1-T_i)}{n_{T_i}}
$$

where $n_{C_i}$ and $n_{T_i}$ are the number of subjects in RCT $i$ in the
control and treatment groups respectively.\\

The mean parameter $\mu$ was given a non-informitive prior $p(\mu) \propto 1$.
This is also the jeffreys prior for the parameter for this sampling
distribution. A non-informative prior was used so as to not influence the
result of the posterior strongly. Though, with a proper prior for $\mu$ with
mean = 0 (a reasonable assumption apriori in a clinical trial) and a variance
of 1 (or smaller, which may still be large for percentages) a similar effect may
be achieved.\\

\subsection{Posterior Distribution for Parameter in $\M_1$}
There is only one unknown quantity in $\M_1$, the mean parameter $\mu$. 
The posterior distribution can be obtained analytically as

$$
\mu | \bm{y} \sim N\p{\frac{\sum_{i=1}^6 y_iV_i^{-1}}{\sum_{i=1}^6V_i^{-1}},
                      \frac{1}{\sum_{i=1}^6V_i^{-1}}}
$$

Figure \ref{fig:m1Post} shows the posterior distribution of $\mu$ in $\M_1$.
Some common statistics are included in the top left corner of the plot. To
answer $\Q$, $p=P\bk{\mu>0|\bm{y},\M_1}$ was computed to be 95.46\%.  A large
value for $p$ favors the aspirin treatment over the control. So, we would
conclude that mortality following a heart attack can be reduced by taking a low
dose of aspirin daily. One could also say that in 95\% of cases, using aspirin
prolonged the lives of post-hear-attack subjects for at least 1 year. (However,
these statements can only be made for people within the demographics of the 6
RCTs.)

\begin{figure}[H]
  \includegraphics[scale=.5]{img/m1Post.pdf}
  \caption{\small Posterior distribution for $\mu$ in $\M_1$.  Posterior mean =
  .0098 (red line). Posterior standard deviation = .0058.  95\% HPD =
  (-.0016,.0212). The darker (navy blue) region is the area of the  distribution
  that is greater than 0, computed to as $P\bk{\mu>0|\bm{y},\M_1}$ = 95.46\%.}
  \label{fig:m1Post}
\end{figure}


\subsection{Models}

$$
\begin{array}{lrcl}
  \M_1: \\
  & y_i | \mu &\sim& N(\mu,V_i)\\
  & p(\mu) &\propto& 1\\
  \\
  \M_2: \\
  & y_i | \theta_i &\sim& N(\theta_i,V_i)\\
  & \theta_i | \mu &\sim& N(\mu,\sigma^2)\\
  & p(\mu) &\propto& 1\\
  & \sigma^2 &\sim& IG(3,.3)\\
\end{array}
$$

$\M_2$ \textbf{is a special case of} $\M_1$. Note that $\M_2$ is
obtained as $\sigma^2 \rightarrow 0$.

\begin{figure}[H]
  \includegraphics[scale=.5]{img/m2MuS2Post.pdf}
  \caption{say...}
  \label{fig:m2MuS2Post}
\end{figure}

\begin{figure}[H]
  \includegraphics[scale=.5]{img/thetaPost.pdf}
  \caption{say...}
  \label{fig:thetaPost}
\end{figure}

\begin{figure}[H]
  \includegraphics[scale=.5]{img/m2MuPost.pdf}
  \caption{say...}
  \label{fig:m2MuPost}
\end{figure}

\section{Analysis}
DIC($\M_2$) - DIC($\M_1$) = .577.
Some analysis

\section{Conclusions}
Some conclusion...

\begin{references}
{\footnotesize
\itemsep=3pt
\item {\em Zellner, Arnold. On assessing prior distributions and Bayesian regression analysis with g-prior distributions. Bayesian inference and decision techniques: Essays in Honor of Bruno De Finetti 6 (1986): 233-243.}
\item {\em Gelman, A., Carlin, J. B., Stern, H. S., \& Rubin, D. B. (2014). Bayesian data analysis (Vol. 2). Boca Raton, FL, USA: Chapman \& Hall/CRC, 73.}
}
\end{references}

\newpage
\section{Source Code for Gibbs Sampler used in $\M_2$}
\verbatiminput{../src/gibbs.R}


\end{document}

%\begin{figure*}
%  \centering
%  \includegraphics[scale=.55]{figs/mapDat.pdf}
%  \vspace{-7em}
%  \caption{\small Some Caption.}
%  \label{fig:mapDat}
%\end{figure*}

%\begin{figure}[H]
%  \includegraphics[scale=.5]{figs/pairsLogRate.pdf}
%  \caption{\small Hi Motor vehicle theft is not strongly correlated with any other thefts.}
%  \label{fig:logOdds}
%\end{figure}

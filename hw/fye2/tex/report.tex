\documentclass{../../tex_template/asaproc}
\usepackage{graphicx} % \includegraphics
\usepackage{float}    % To keep figures in right place. 
                      % Usage: \being{figure}[H] \includegraphics{tmp.pdf} \end{figure}
\usepackage{subfig}   % \subfloat
\usepackage{amsmath}  % bmatrix, pmatrix, etc
\usepackage{amsfonts} % \mathbb{Q}, etc.
\usepackage{bm}
\newcommand{\p}[1]{\left(#1\right)}
\newcommand{\bk}[1]{\left[#1\right]}
\newcommand{\bc}[1]{ \left\{#1\right\} }
\newcommand{\abs}[1]{ \left|#1\right| }
\newcommand{\norm}[1]{ \left|\left|#1\right|\right| }
\newcommand{\E}{ \text{E} }
\newcommand{\N}{ \mathcal N }
\newcommand{\ds}{ \displaystyle }

%\usepackage{times}
%If you have times installed on your system, please
%uncomment the line above

%For figures and tables to stretch across two columns
%use \begin{figure*} \end{figure*} and
%\begin{table*}\end{table*}
% please place figures & tables as close as possible
% to text references

\newcommand{\be}{\begin{equation}}
\newcommand{\ee}{\end{equation}}
\newcommand{\y}{\bm y}
\newcommand{\M}{\mathcal{M}}
\usepackage{verbatim}
\newcommand{\Q}{\mathbb{Q}}
\newcommand{\simi}{\overset{ind.}{\sim}}
\newcommand{\iid}{\overset{iid}{\sim}}
\newcommand{\sumk}{\sum_{i=1}^6}

\title{FYE--- Aspirin}

%input all authors' names
\author{
  Test ID: 911$^1$\\
  University California -- Santa Cruz$^1$\\
}

%input affiliations
%{USDA Forest Service Forest Products Laboratory}

\begin{document}
\maketitle
\begin{abstract}
$\Q$: \textbf{Can mortality following a heart attack be reduced by taking a low
dose of aspirin daily?} Using the data from six randomized controlled trials
(RCTs), and two models ($\M_1$ -- fixed-effects model and $\M_2$ --
random-effects model) the question ($\Q$) was studied. The following is a
summary of the findings. (a) Visually, it seems that aspirin had a consistent
effect that is large in clinical terms in most of the RCTs (see Figure
\ref{fig:tcy}).  (b) The probability that the control groups' mortality rate is
higher than that of the treatment groups across all six studies computed using
a fixed effects model and given the provided data to be $P\bk{\mu > 0 | y, \M_1} =
95.46\%$. This suggests that under $\M_1$, the mortality rate for patients (in
the six RCTs) who take low doses of aspirin following a heart attack is reduced
in most cases (95\% of the time). (c)(i) $\M_1$ is a special case of $\M_2$.
Specifically, when $\mu$ has the same prior in both models and $\sigma^2
\rightarrow 0$ in $\M_2$, $\M_2$ becomes $\M_1$. (i.e. if the $\theta_i$'s are
all the same in $\M_2$, then $\M_1$ is retrieved.) A simple way to decide if
$\M_2$ is better than $\M_1$ is to compare the DICs of the two models. The
model with the smaller DIC is preferred. Another way to determine which model
is better is to compute the posterior probability of each model.  $\M_2$ is
favored as it has a larger model posterior probability. (c)(ii) The probability
that the control groups' mortality rate is higher than that of the treatment
groups across all six studies computed using a random effects model and given
the provided data to be $P\bk{\mu > 0 | y, \M_2} = 56.78\%$. This suggests that
under $\M_2$, the mortality rate for patients (in the six RCTs) who take low
doses of aspirin following a heart attack is reduced in 56.78\% of the cases.
(The full conditionals for each parameter in $\M_2$ are included in this
paper.)

\begin{keywords}
Random effects vs fixed effects models, aspirin, heart attack.
\end{keywords}
\end{abstract}

\section{Introduction}
People with heart problems commonly use aspirin to reduce blood pressure and
the chances of a heart attack. In the early 1990's when aspirin was known to
have a blood-thinning effect. But it was not known if the drug saved lives.  A
dataset containing the data of 6 randomized controlled trials (RCTs) were used
to analyze the effectiveness of aspirin in reducing mortality after a heart
attack. The 6 studies included were UK-1, CDPA, GAMS, UK-2, PARIS, and AMIS.
The data included the number patients in the studies (both control and
treatment groups), as well as the all-cause mortality rates for both groups.
The control and treatment groups were divided fairly evenly. But the number of
patients were not all equal. Particularly, the number of patients in the AMIS
is more than double that of any other study. This potentially could affect the
results of any meta-analysis if not accounted for appropriately. Two models
will be used to do meta-analysis on the 6 studies: (1) a fixed-effects model
and (2) a random-effects model. The order of the material in this paper will
proceed as follows. First, an exploratory analysis of the data will be done to
show obvious trends and patterns in the data, as well as motivate the need for
two different models. Second, the two models will be presented. The priors,
posterior distributions and complete conditionals for all parameters will also
be presented. Finally, some concluding remarks will be made.

\section{Exploratory Analysis}
To better understand the data, I first visualized it.  Figure \ref{fig:tcy}
shows, for each of the six studies, the mortality rates of the control groups
(red) and treatment (blue) groups, along with the difference in mortality rates
$y_i = C_i-T_i$ (green).  \textbf{Across five of the six studies (i.e. most of
the studies), the mortality rates of the control group are higher than that of
the treatment group.} (Notice how the red line is usually higher than the blue
line by an approximately constant offset, and the green line is usually above
0.) The only exception is the AMIS study group -- the control group has a
mortality rate 9.7, which is slightly less than that of the treatment group
(10.85).  It certainly \textbf{seems that aspirin had an effect on most of the
RCTs}. More investigation will be required to understand why the mortality
rates of the control group were lower than that of the treatment group in the
AMIS study, especially since the AMIS study had the most participants -- at
least double that of any other study. It may be of worth to researchers to
investigate the reasons for such a difference. Possible reasons could be
related to the demography and environment of the location where the AMIS study
was conducted.  I would say the \textbf{effect of aspirin appears to be great
in clinical terms in most of the RCTs} because for the studies where aspirin
seemed to have a positive effect, the mortality rates of the treatment group
were 2.4\% lower than that of the control group on average. While 2.4\% may not
seem like a large portion in other fields, it should be kept in mind that the
proportion refers to proportion of lives saved. And the life of every person is
of great value.\\

\begin{figure}%[H]
  \includegraphics[scale=.5]{img/tcy.pdf}
  \caption{\small Mortality rates for control (red) and treatment (blue) groups
  for different randomized control trials. The green line shows the difference
  in mortality rates between the control group and the treatment group (control -
  treatment). A lower mortality rate is better.}
  \label{fig:tcy}
\end{figure}

It should be expected that different study groups will have different mortality
rates, due to the different demographics of participants in each study. But the
fact that the mortality rates are lower than that of the treatment group by a
more-or-less constant amount may be indicating that aspirin has a constant or
fixed effect on mortality rates.  More formal statistical analysis will be
needed to make more informative statements.\\

This concludes the exploratory analysis. In the next section, I describe the
statistical models used to study the data further.\\

\section{Models} 
To answer the question ($\Q$) of interest, two models were fit to the data --
(1) a fixed-effects model and (2) a random-effects model. I will provide the
model and prior specifications, as wells as the full conditionals and posterior
distributions for each parameter in this section.\\

\subsection{Fixed-effects Model}
First, a fixed-effects model was used to combine the information from the six
(similar but different) studies. This model assumes no between-experiment
heterogeneity. (i.e. effects are assumed to be constant across trials.) From
Figure \ref{fig:tcy}, one could reason that homogeneity between experiments is not
a good assumption.  The difference in study groups will be addressed with
random-effects models in the next section. I will refer to the fixed-effects
model as $\M_1$, which has the following form:

$$
\begin{array}{lrcl}
  \M_1: \\
  & y_i | \mu &\simi& N(\mu,V_i),~\text{for$~i = 1,...6$}\\
  & p(\mu) &\propto& 1\\
\end{array}
$$

where $y_i = (C_i - T_i)$, $C_i$ is the mortality rate in the control group for
study $i$, and $T_i$ is the mortality rate in the treatment group for study
$i$. Note that if $y_i > 0$ then the mortality rate of study $i$ is higher
for the control group, suggesting that the treatment is reducing the
mortality rates of subjects in the study.\\

In this model, the sampling distribution for each difference ($y_i$) is assumed
to be normal with mean $\mu$ (shared across all RCTs -- hence fixed-effects)
and individual (binomial) variances $V_i$. The variances are treated as known
since the sample sizes in each RCT are large enough. Specifically,

$$
V_i = \frac{C_i(1-C_i)}{n_{C_i}} + \frac{T_i(1-T_i)}{n_{T_i}}
$$

where $n_{C_i}$ and $n_{T_i}$ are the number of subjects in RCT $i$ in the
control and treatment groups respectively.\\

The mean parameter $\mu$ was given a non-informative prior $p(\mu) \propto 1$.
This is also the Jeffreys prior for the parameter for this sampling
distribution. A non-informative prior was used so as to not influence the
result of the posterior strongly. Though, with a proper prior for $\mu$ with
mean = 0 (a reasonable assumption apriori in a clinical trial) and a variance
of 1 (or smaller, which may still be large for percentages) a similar effect may
be achieved.\\

\subsection{Posterior Distribution for Parameter in $\M_1$}
There is only one unknown quantity in $\M_1$, the mean parameter $\mu$. 
The posterior distribution can be obtained analytically as

$$
\mu | \bm{y} \sim N\p{\frac{\sum_{i=1}^6 y_iV_i^{-1}}{\sum_{i=1}^6V_i^{-1}},
                      \frac{1}{\sum_{i=1}^6V_i^{-1}}}
$$

Figure \ref{fig:m1Post} shows the posterior distribution of $\mu$ in $\M_1$.
The posterior mean is 0.0098, the standard deviation is 0.0058, and the 95\%
HPD is (-.0016,.0212).  To answer $\Q$, $p=P\bk{\mu>0|\bm{y},\M_1}$ was
computed to be 95.46\%.  (These statistics are included in the top left corner
Figure \ref{fig:m1Post}.) A large value for $p$ favors the aspirin treatment
over the control.  So, we would conclude that mortality following a heart
attack can be reduced by taking a low dose of aspirin daily. One could also say
that in 95\% of cases, using aspirin prolonged the lives of post-heart-attack
subjects for at least 1 year.
%(However, care must be taken when attempting to extrapolate the results to a
%more general population.)

\begin{figure}%[H]
  \includegraphics[scale=.5]{img/m1Post.pdf}
  \caption{\small Posterior distribution for $\mu$ in $\M_1$.  Posterior mean =
  .0098 (red line). Posterior standard deviation = .0058.  95\% HPD =
  (-.0016,.0212). The darker (navy blue) region is the area of the  distribution
  that is greater than 0, computed to as $P\bk{\mu>0|\bm{y},\M_1}$ = 95.46\%.}
  \label{fig:m1Post}
\end{figure}

\subsection{Random-effects Model}
As mentioned in the previous section. The fixed-effects model assumes no
between-experiment heterogeneity. This assumption may be too strong for this
analysis since the AMIS study yielded different results than the other 5
studies. The random-effects model allows for between-experiment heterogeneity
while still combining information from different studies. A Gaussian version
of such a model, which will be called $\M_2$, is as follow:

$$
\begin{array}{lrcl}
  \M_2: \\
  & y_i | \theta_i &\simi& N(\theta_i,V_i)\\
  & \theta_i | \mu &\iid& N(\mu,\sigma^2)\\
  & p(\mu) &\propto& 1\\
  & \sigma^2 &\sim& IG(a,b)\\
\end{array}
$$
where $a = 2$, and $b = 1/3$.\\

In this model, each $y_i$ is modelled by a normal distribution with a unique
mean $\theta_i$ and variance $V_i$. The $\theta_i$'s are independently and
identically distributed as normal distributions with mean $\mu$ and variance
$\sigma^2$. Both of these unknown quantities are further modelled with
\textbf{independent prior distributions} $p(\mu)\propto 1$ (Jeffreys prior) and
an Inverse-Gamma distribution. A Jeffreys prior was used for $\mu$ for similar
reasons to the subsection above -- so as to remain objective. The prior for
$\sigma^2$ was chosen so that the prior mean is 1/3 and the prior variance is
$\infty$. The prior mean of 1/3 is approximately the variance of one random
Uniform(-1,1) variable. That is apriori, I expect the variance of each
$\theta_i$ to be the variance of a random Uniform(-1,1) variable. \\

Note that the priors used are \textbf{conjugate}.\\

Also, note that $\M_2$ \textbf{is a special case of} $\M_1$. $\M_2$ is
obtained as $\sigma^2 \rightarrow 0$. As $\sigma^2 \rightarrow 0$, 
the $\theta_i$'s will be centered at $\mu$ with 0 variance. This
is simply the fixed-effects model.\\

\subsection{Posterior Distribution for Parameters in $\M_2$}
There are 8 parameters in $\M_2$. While the posterior distribution of the
parameters cannot be analytically computed jointly, each of their complete
conditionals can be computed in closed form. They are as follows:

\newcommand{\ivy}{\sumk V_i^{-1}}
$$
\begin{array}{rcl}
  \mu | \bm y,\bm\theta,\sigma^2  &\sim& 
  N\p{\ds\frac{\sumk \theta_i}{6},\ds\frac{\sigma^2}{6}}\\
  \sigma^2 | \bm y,\bm\theta , \mu &\sim& 
  IG\p{a+6/2,\ds\frac{\sumk(\theta_i-\mu)^2}{2\sigma^2}+b}\\
  \theta_i | \mu, \sigma^2, \bm y &\sim& 
  N\p{\ds\frac{y_iV_i^{-1}+\mu/\sigma^2}{V_i^{-1}+1/\sigma^2},\p{V_i^{-1}+1/\sigma^2}^{-1}},\\
  &&\hspace{10em} \text{~for~} i=1,...,6\\
\end{array}
$$

The complete conditionals were used in a Gibbs sampler to efficiently update
each parameter. For each parameter, 10,000 samples were obtained after a
burn-in period of 5000.  The posterior distribution for each parameter are
shown below. Figure \ref{fig:m2MuS2Post} shows the posterior distributions for
$\mu$ and $\sigma^2$. Figure \ref{fig:thetaPost} shows the posterior
distribution for $\bm\theta$. To supplement the busy figure, Table
\ref{tab:thetaPost} lists the posterior means, standard deviations and 95\%
HPDs for each $\theta_i$. I will only discuss the posterior distributions for
$\mu$ and $\sigma^2$. Refer to the figures and table for more information about
the posterior distributions.\\

The posterior distribution for $\mu$ has mean = 0.02056, standard deviation
= 0.128, and 95\% HPD = $(-0.226,0.2774)$. In addition, the trace plots
do not show evidence of non-convergence.\\

The posterior distribution for $\sigma^2$ has mean = 0.09628, standard deviation
= 0.059, and 95\% HPD = $(0.0241,0.2027)$. Again, the trace plots
do not show evidence of non-convergence.\\

\begin{figure}[H]
  \includegraphics[scale=.5]{img/m2MuS2Post.pdf}
  \caption{Posterior distributions for $\mu$ and $\sigma^2$. Posterior
  mean, SD, and 95\% HPDs are included in the top left corner of the
  univariate plots. Trace plots on top right corner of univariate plots. The
  bottom-left plot shows the correlation between the two random variables. The
  upper corner shows the bivariate trace plot and posterior distribution
  for ($\mu,\sigma^2$).}
  \label{fig:m2MuS2Post}
\end{figure}

\begin{figure}[H]
  \includegraphics[scale=.5]{img/thetaPost.pdf}
  \caption{Posterior distributions for $\bm\theta$. Correlation of each pair of
  variables in lower triangle subplots. Bivariate distribution on upper triangle
  subplots. The posteriors appear to be independent and appear to have converged.}
  \label{fig:thetaPost}
\end{figure}

%% latex table generated in R 3.3.1 by xtable 1.8-2 package
% Thu Sep 15 19:52:19 2016
\begin{table}[ht]
\centering
\begin{tabular}{rrrr}
  \hline
 & Mean & SD & 95\% HPD \\ 
  \hline
  $\theta_1$ & 0.027 & 0.017 & (-0.005,~ 0.059) \\ 
  $\theta_2$ & 0.025 & 0.013 & (-0.001,~ 0.051) \\ 
  $\theta_3$ & 0.019 & 0.024 & (-0.029,~ 0.063) \\ 
  $\theta_4$ & 0.025 & 0.017 & (-0.007,~ 0.060) \\ 
  $\theta_5$ & 0.023 & 0.020 & (-0.017,~ 0.059) \\ 
  $\theta_6$ &-0.011 & 0.009 & (-0.029,~ 0.006) \\ 
   \hline
\end{tabular}
\caption{hi}
\label{tab:thPost}
\end{table}

\begin{table}[ht]
\centering
\begin{tabular}{rrrr}
  \hline
 & Mean & SD & 95\% HPD \\ 
  \hline
  $\theta_1$ & 0.027 & 0.017 & (-0.005,~ 0.059) \\ 
  $\theta_2$ & 0.025 & 0.013 & (-0.001,~ 0.051) \\ 
  $\theta_3$ & 0.019 & 0.024 & (-0.029,~ 0.063) \\ 
  $\theta_4$ & 0.025 & 0.017 & (-0.007,~ 0.060) \\ 
  $\theta_5$ & 0.023 & 0.020 & (-0.017,~ 0.059) \\ 
  $\theta_6$ &-0.011 & 0.009 & (-0.029,~ 0.006) \\ 
   \hline
\end{tabular}
\caption{Posterior mean, standard deviation, and 95\% HPD for
$\theta_1,...\theta_6$.}
\label{tab:thetaPost}
\end{table}

We again answer $\Q$ with the posterior distribution of $\mu$ in $\M_2$.  The
posterior distribution of $\mu$ is again plotted in Figure \ref{fig:m2MuPost}.
The figure highlights the area under the density which is greater than 0, in
navy blue. $P\bk{\mu>0|\bm{y},\M_2}$ \textbf{was computed to be 56.78\%}. (This
is much smaller than 95\% in the fixed-effects model.) One could say that in
56\% of cases (for people in the demographics of the RCTs), using aspirin
prolonged the lives of post-heart-attack subjects for at least 1 year.  We
conclude that mortality following a heart attack is reduced \textbf{only 56\%}
of the time by taking a low dose of aspirin daily. In other words,
\textbf{mortality is not significantly reduced by taking a low dose of
aspirin daily, under} $\M_2$.

\begin{figure}[H]
  \includegraphics[scale=.5]{img/m2MuPost.pdf}
  \caption{ Posterior distribution for $\mu$ in $\M_2$. Posterior mean = 0.0266
    (red line). SD = 0.1277. 95\% HPD = (-0.226,0.2774). The darker (navy
    blue) region is the area of the distribution that is greater than 0,
    computed to as $P\bk{\mu>0|\bm{y},\M_2}$ = 56.78\%.}
  \label{fig:m2MuPost}
\end{figure}

\subsection{How to Compare $\M_1$ and $\M_2$}
The two models can be compared using deviance information criterion (DIC).
DIC is defined as 

$$
DIC = \bar{D}(\theta) + \widehat{\text{var}}\p{D(\theta)} / 2
$$

where $D(\theta)= -2\log p(\bm y|\theta)$, $\bar{D}(\theta)$ is the average of
$D(\theta)$ over all posterior samples $\theta$, and
$\widehat{\text{var}}(D(\theta))$ is the sample variance of $D(\theta)$. DIC
measures goodness-of-fit and penalizes model complexity; a lower DIC is
preferred. \\

DIC($\M_2$) - DIC($\M_1$) = 0.5632. This suggests that $\M_1$ is slightly
``better" than $\M_2$. i.e. $\M_1$ models the data well and uses fewer
parameters.\\

DIC is a simple tool for model selection. Other methods may be preferred 
in this case to determine which model is more fitting. In particular,
to answer the question $\Q$, computing the posterior probability
for $\M_i$

$$
P\bk{\M_i|\bm y} = \ds\frac{p(\bm y|\M_i)}{p(\bm y|\M_1)+p(\bm y|\M_2)}
$$

may be more useful as it tells us which model is more likely given that the
models are evenly weighted apriori. A larger posterior probability is
preferred. $P\bk{\M_1|\bm y}$ was computed to be 17\% and $P\bk{\M_2|\bm y}$
was computed to be 82\%. So, $\M_2$ \textbf{is favored using this method}.\\


\section{Conclusions}
To conclude, though in an exploratory analysis it may appear that aspirin has a
consistent effect in most of the RCTs, a formal statistical analysis may say
otherwise. The fixed-effects model yielded $P\bk{\mu > 0 | y, \M_1} = 95.46\%$
-- which leads one to believe that taking aspirin after a heart attack reduces
mortality rate. The random-effects model yielded $P\bk{\mu > 0 | y, \M_2} = 56.78\%$
-- which leads one to believe that taking aspirin after a heart attack does
not significantly reduce mortality. The random-effects model is preferred here
based on its model posterior probability. However, when using DIC as a
modelling criterion, the fixed effects model is preferred. Furthermore,
investigating why the results of the AMIS study varied so much from the
that of the other studies may be of interest to health-care professional
and those interested in the effects of aspirin.

%\begin{references}
%{\footnotesize
%\itemsep=3pt
%\item {\em Zellner, Arnold. On assessing prior distributions and Bayesian regression analysis with g-prior distributions. Bayesian inference and decision techniques: Essays in Honor of Bruno De Finetti 6 (1986): 233-243.}
%\item {\em Gelman, A., Carlin, J. B., Stern, H. S., \& Rubin, D. B. (2014). Bayesian data analysis (Vol. 2). Boca Raton, FL, USA: Chapman \& Hall/CRC, 73.}
%}
%\end{references}

\newpage
\section{Source Code for Gibbs Sampler used in $\M_2$}
\verbatiminput{../src/gibbs.R}


\end{document}

%\begin{figure*}
%  \centering
%  \includegraphics[scale=.55]{figs/mapDat.pdf}
%  \vspace{-7em}
%  \caption{\small Some Caption.}
%  \label{fig:mapDat}
%\end{figure*}

%\begin{figure}[H]
%  \includegraphics[scale=.5]{figs/pairsLogRate.pdf}
%  \caption{\small Hi Motor vehicle theft is not strongly correlated with any other thefts.}
%  \label{fig:logOdds}
%\end{figure}

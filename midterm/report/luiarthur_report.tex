\documentclass{../../tex_template/asaproc}
\usepackage{graphicx} % \includegraphics
\usepackage{float}    % To keep figures in right place. 
                      % Usage: \being{figure}[H] \includegraphics{tmp.pdf} \end{figure}
\usepackage{subfig}   % \subfloat
\usepackage{amsmath}  % bmatrix, pmatrix, etc
\usepackage{bm}
\newcommand{\p}[1]{\left(#1\right)}
\newcommand{\bk}[1]{\left[#1\right]}
\newcommand{\bc}[1]{ \left\{#1\right\} }
\newcommand{\abs}[1]{ \left|#1\right| }
\newcommand{\E}{ \text{E} }
\newcommand{\Y}{ \bm Y }
\newcommand{\ds}{ \displaystyle }

%\usepackage{times}
%If you have times installed on your system, please
%uncomment the line above

%For figures and tables to stretch across two columns
%use \begin{figure*} \end{figure*} and
%\begin{table*}\end{table*}
% please place figures & tables as close as possible
% to text references

\newcommand{\be}{\begin{equation}}
\newcommand{\ee}{\end{equation}}

\title{Quiz 1 --- California County Thefts}

%input all authors' names

\author{
  Arthur Lui$^1$\\
  University California -- Santa Cruz$^1$\\
}

%input affiliations

%{USDA Forest Service Forest Products Laboratory}

\begin{document}

\maketitle


\begin{abstract}
Abstract goes here.
\begin{keywords}
Keywords go here
\end{keywords}
\end{abstract}


\section{Introduction}
\section{Methods}
\section{Analysis}
\section{Conclusions}

\begin{references}
{\footnotesize
\itemsep=3pt

\item {\em Gelman, A., Carlin, J. B., Stern, H. S., \& Rubin, D. B. (2014). Bayesian data analysis (Vol. 2). Boca Raton, FL, USA: Chapman \& Hall/CRC, 73.}

}
\end{references}
\end{document}

%\begin{figure*}
%  \centering
%  \includegraphics[scale=.55]{figs/mapDat.pdf}
%  \vspace{-7em}
%  \caption{\small Some Caption.}
%  \label{fig:mapDat}
%\end{figure*}


\documentclass{../../tex_template/asaproc}
\usepackage{graphicx} % \includegraphics
\usepackage{float}    % To keep figures in right place. 
                      % Usage: \being{figure}[H] \includegraphics{tmp.pdf} \end{figure}
\usepackage{subfig}   % \subfloat
\usepackage{amsmath}  % bmatrix, pmatrix, etc
\usepackage{bm}
\newcommand{\p}[1]{\left(#1\right)}
\newcommand{\bk}[1]{\left[#1\right]}
\newcommand{\bc}[1]{ \left\{#1\right\} }
\newcommand{\abs}[1]{ \left|#1\right| }
\newcommand{\E}{ \text{E} }

%\usepackage{times}
%If you have times installed on your system, please
%uncomment the line above

%For figures and tables to stretch across two columns
%use \begin{figure*} \end{figure*} and
%\begin{table*}\end{table*}
% please place figures & tables as close as possible
% to text references

% To combine plots:
%\begin{figure*}
%  \subfloat[][]{\includegraphics[scale=.22]{figs/mapDatRob.pdf}}\quad
%  \subfloat[][]{\includegraphics[scale=.22]{figs/mapDatRob.pdf}}\quad
%  \subfloat[][]{\includegraphics[scale=.22]{figs/mapDatRob.pdf}}\quad
%  \subfloat[][]{\includegraphics[scale=.22]{figs/mapDatRob.pdf}}\quad
%\end{figure*}

\newcommand{\be}{\begin{equation}}
\newcommand{\ee}{\end{equation}}

\title{Quiz 1 --- California County Thefts}

%input all authors' names

\author{
  Arthur Lui$^1$\\
  University California -- Santa Cruz$^1$\\
}

%input affiliations

%{USDA Forest Service Forest Products Laboratory}

\begin{document}

\maketitle


\begin{abstract}
 This abstract should contain a short description of the problem and the main findings.
\begin{keywords}
Please place 3--5 key words here.
\end{keywords}
\end{abstract}


\section{Introduction}
This intro should also contain a description of the problem and an exploratory data analysis.

\begin{figure}[H]
  \includegraphics[scale=.5]{figs/pairsLogRate.pdf}
  \label{fig:logRate}
  \caption{\small Histogram for logged number of each theft per capita on diagonals. Scatter plots
  of each pair of variables in upper triangle. Correlation between each pair of variables in
  lower triangle. All correlations are positive. Larceny and burglary are highly correlated (.93). 
  Motor vehicle theft is not strongly correlated with any other thefts.}
\end{figure}

Words words words.  Words words words.  Words words words.  Words words words.  Words words words.
Words words words.  Words words words.  Words words words.  Words words words.  Words words words.
Words words words.  Words words words.  Words words words.  Words words words.  Words words words.
Words words words.  Words words words.  Words words words.  Words words words.  Words words words.
Words words words.  Words words words.  Words words words.  Words words words.  Words words words.
Words words words.  Words words words.  Words words words.  Words words words.  Words words words.
Words words words.  Words words words.  Words words words.  Words words words.  Words words words.
Words words words.  Words words words.  Words words words.  Words words words.  Words words words.
Words words words.  Words words words.  Words words words.  Words words words.  Words words words.
Words words words.  Words words words.  Words words words.  Words words words.  Words words words.
\begin{figure*}
  \centering
  \includegraphics[scale=.55]{figs/mapDat.pdf}
  \vspace{-7em}
  \caption{Map of California thefts per 100,000 people for a selected of counties. The left-most plot
  shows the number of robberies per 100,100 for each of the counties. The color code in the shows that
greener areas have fewer robberies (safer) and redder areas have more robberies (more dangerous).}
  \label{fig:mapDat}
\end{figure*}


Words words words.  Words words words.  Words words words.  Words words words.  Words words words.
Words words words.  Words words words.  Words words words.  Words words words.  Words words words.
Words words words.  Words words words.  Words words words.  Words words words.  Words words words.
Words words words.  Words words words.  Words words words.  Words words words.  Words words words.
Words words words.  Words words words.  Words words words.  Words words words.  Words words words.
Words words words.  Words words words.  Words words words.  Words words words.  Words words words.
Words words words.  Words words words.  Words words words.  Words words words.  Words words words.
Words words words.  Words words words.  Words words words.  Words words words.  Words words words.
Words words words.  Words words words.  Words words words.  Words words words.  Words words words.
Words words words.  Words words words.  Words words words.  Words words words.  Words words words.

\section{Methods}
This section contains the methods. Normal Inverse Wisahrt.
Words words words.  Words words words.  Words words words.  Words words words.  Words words words.
Words words words.  Words words words.  Words words words.  Words words words.  Words words words.
Words words words.  Words words words.  Words words words.  Words words words.  Words words words.
Words words words.  Words words words.  Words words words.  Words words words.  Words words words.
Words words words.  Words words words.  Words words words.  Words words words.  Words words words.
Words words words.  Words words words.  Words words words.  Words words words.  Words words words.
Words words words.  Words words words.  Words words words.  Words words words.  Words words words.
Words words words.  Words words words.  Words words words.  Words words words.  Words words words.
Words words words.  Words words words.  Words words words.  Words words words.  Words words words.

\section{Analysis}
This part is the actual analysis and results.
Words words words.  Words words words.  Words words words.  Words words words.  Words words words.
\begin{figure}
  \centering
  \includegraphics[scale=.3]{figs/postMu.pdf}
  \caption{Posterior distribution of the mean parameter.}
  \label{fig:postMu}
\end{figure}


Words words words.  Words words words.  Words words words.  Words words words.  Words words words.
Words words words.  Words words words.  Words words words.  Words words words.  Words words words.
\[
\begin{array}{lll}
  \E[\bm\Sigma | \bm Y] &=& \p{\input{figs/postmeanS.tex}} \\
    \\
  (\bm\Sigma_{2.5\%}| \bm Y) &=& \p{\input{figs/postLoS.tex}} \\
    \\
    (\bm\Sigma_{97.5\%} | \bm Y) &=& \p{\input{figs/postHiS.tex}} \\
\end{array}
\]



Words words words.  Words words words.  Words words words.  Words words words.  Words words words.
Words words words.  Words words words.  Words words words.  Words words words.  Words words words.
Words words words.  Words words words.  Words words words.  Words words words.  Words words words.
Words words words.  Words words words.  Words words words.  Words words words.  Words words words.
Words words words.  Words words words.  Words words words.  Words words words.  Words words words.
Words words words.  Words words words.  Words words words.  Words words words.  Words words words.




\end{document}


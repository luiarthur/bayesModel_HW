\documentclass{../../tex_template/asaproc}
\usepackage{graphicx} % \includegraphics
\usepackage{float}    % To keep figures in right place. 
                      % Usage: \being{figure}[H] \includegraphics{tmp.pdf} \end{figure}
\usepackage{subfig}   % \subfloat
\usepackage{amsmath}  % bmatrix, pmatrix, etc
\usepackage{bm}
\newcommand{\p}[1]{\left(#1\right)}
\newcommand{\bk}[1]{\left[#1\right]}
\newcommand{\bc}[1]{ \left\{#1\right\} }
\newcommand{\abs}[1]{ \left|#1\right| }
\newcommand{\E}{ \text{E} }
\newcommand{\Y}{ \bm Y }
\newcommand{\ds}{ \displaystyle }

%\usepackage{times}
%If you have times installed on your system, please
%uncomment the line above

%For figures and tables to stretch across two columns
%use \begin{figure*} \end{figure*} and
%\begin{table*}\end{table*}
% please place figures & tables as close as possible
% to text references

\newcommand{\be}{\begin{equation}}
\newcommand{\ee}{\end{equation}}

\title{Quiz 1 --- California County Thefts}

%input all authors' names

\author{
  Arthur Lui$^1$\\
  University California -- Santa Cruz$^1$\\
}

%input affiliations

%{USDA Forest Service Forest Products Laboratory}

\begin{document}

\maketitle


\begin{abstract}
% This abstract should contain a short description of the problem and the main findings.
Given a data set containing the number of 4 seperate categories of thefts in
California counties, we would like to model the expected number of each theft
in each county. We would also like to know the probability that the total
number of thefts in Santa Cruz county exceeds 3000. A multivariate Normal
sampling distribution was placed on the logit of the proportion of each theft
per each county. A Normal Inverse Wishart prior was placed on the mean vector
and covariance matrix respectively. The posterior distribution of the parameters
are explored in this paper. The posterior probability that the number of thefts 
in Santa Cruz exceeds 3000 was computed to be 12.8\%.
\begin{keywords}
Normal likelihood, Normal-Inverse-Wishart prior.
\end{keywords}
\end{abstract}


\section{Introduction}
The data provided contains 6 columns: a column containing the names of 39
counties in California, a column indicating the population of each of the
counties, and 4 columns consisting counts of 4 types of thefts (I) robbery,
(II) burglary, (III) larceny, and (Iv) motor vehicle theft. The goal is to
answer some questions about the pattern of theft in the state. Some exploratory
analysis of the data shows that the number of each theft is highly correlated.
This is not surprising because the numbers of different thefts are confounded
by population sizes of the counties. That is, if a county (say Los Angeles) has
a larger population than another (say Santa Cruz), then the number of all
thefts are going to be high (simply because there are more people and hence
more crimes). To construct a somewhat sensible model, the number of thefts per
county needs first to be transformed to the number of crimes per capita in each
county. Still, since crime rates tend to have a small order of magnitude, some
kind of transfomation on a log scale is needed to make computation more
reasonable. The log of the crimes per capita seemed linear. We could model
these log probabilities with a multivariate normal sampling distribution. But,
modeling the log odds would be more sensible because log odds truly have infinite
support; while log probabilities only have negative support. Figure \ref{fig:logOdds}
shows the log odds of each theft per county. The diagonals are the histograms for 
each theft. The upper traingle shows the scatter plots of the logged odds for each 
pair of thefts. The lower triangle shows the correlation. The logged odds
are somewhat linear. Vehicle thefts do not seem to be very much correlated to any
other theft. Larceny and burglary are highly correlated (.93).

\begin{figure}[H]
  \includegraphics[scale=.5]{figs/pairsLogRate.pdf}
  \caption{\small Histogram for logit of number of each theft per capita on diagonals. Scatter plots
  of each pair of variables in upper triangle. Correlation between each pair of variables in
  lower triangle. All correlations are positive. Larceny and burglary are highly correlated (.93). 
  Motor vehicle theft is not strongly correlated with any other thefts.}
  \label{fig:logOdds}
\end{figure}

To provide additional intuition for the data, Figure \ref{fig:mapDat} displays the 
the number of thefts per 100,000 people for each county studied. The counties are 
colored in a way such that areas that are greener having fewer thefts; and areas
that are more red have more thefts. Each color corresponds to a bin of numbers
which are simply the 20, 40, 60, and 80 $^{th}$ percentiles of the data.
Larceny and burglary are the most common thefts. Vehicle thefts and robberies
are not as common. 

\begin{figure*}
  \centering
  \includegraphics[scale=.55]{figs/mapDat.pdf}
  \vspace{-7em}
  \caption{\small Map of California thefts per 100,000 people for a selection of
  counties. The left-most plot shows the number of thefts per 100,000 people for
  each of the counties. The color code shows that greener areas have fewer
  robberies (safer) and redder areas have more robberies (more dangerous). The
  different color bins also represent the 20, 40, 60, and 80$^{th}$ percentile
  cut-offs.}
  \label{fig:mapDat}
\end{figure*}


\section{Methods}
The log odds can be modeled with a Normal distribution with dimension 4.  Let
$\bm Y$ be a $39\times 4$ matrix where each row represents a county and each
column represents a theft category. $\bm Y_{i,j}$ will denote the logit of the
number of theft $j$ per population size in county $i$. Moreover, let $\bm Y_i$
denote the $i^{th}$ row of matrix $\bm Y$, and let $\bar{\bm Y}$ be the 4
column averages of the matrix.  Then,
\[
  \Y_i^T | \bm{\mu,\Sigma} \sim \mathcal{N}_4(\bm{\mu,\Sigma})
\]
for $i=1,...,n$ where $n=39$ is the sampling model. A Normal-Inverse-Wishart prior is
placed on $(\bm{\mu,\Sigma})$. This simplifies computation as the prior is
conjugate and the posterior can be directly sampled from. Specifially, if
$\bm\mu | \bm\Sigma \sim \mathcal{N}(\bm{m,\Sigma}/\kappa)$ and $\bm \Sigma
\sim \mathcal{IW}(v,\bm S^{-1})$, we can sample from the posterior by
first drawing $\bm \Sigma | \bm Y \sim \mathcal{IW}(v_n,\bm S_n^{-1})$, then 
drawing $\bm{\mu | \Sigma, Y} \sim \mathcal{N}(\bm{m_n,\Sigma}/\kappa_n)$.
The values of the updated parameters are
\[
\begin{array}{rcl}
  \bm m_n &=& \ds\frac{\kappa\bm m + n\bar{\bm Y}}{\kappa+n} \\
  \kappa_n &=& \kappa + n \\
  v_n &=& v + n \\
  \bm S_n &=& \bm{S + C} + \ds\frac{\lambda n}{\lambda+n} 
  (\boldsymbol{\bm{\bar{Y}-m}})^T(\boldsymbol{\bm{\bar{Y}-m}}) \\
  \\
  \text{where } \boldsymbol{C} &=& \sum_{i=1}^{n} (\boldsymbol{y_i-\bar{y}})^T(\boldsymbol{y_i-\bar{y}}) \\
\end{array}
\]
To sample from the posterior predictive distribution, we can simply draw
$\tilde{\bm Y} | \bm{\mu,\Sigma,Y} \sim \mathcal{N}(\bm{\mu,\Sigma})$ using the newly drawn
values for $\bm\mu$ and $\bm\Sigma$.

In this analysis, $(\bm m, \kappa, v, \bm S) = ()$



\section{Analysis}
This part is the actual analysis and results.
Words words words.  Words words words.  Words words words.  Words words words.  Words words words.
\begin{figure}
  \centering
  \includegraphics[scale=.2]{figs/postMu.pdf}
  \caption{Posterior distribution of the mean parameter.}
  \label{fig:postMu}
\end{figure}


Words words words.  Words words words.  Words words words.  Words words words.  Words words words.
Words words words.  Words words words.  Words words words.  Words words words.  Words words words.
\[
\begin{array}{lll}
  \E[\bm\Sigma | \bm Y] &=& \p{\input{figs/postmeanS.tex}} \\
    \\
  (\bm\Sigma_{2.5\%}| \bm Y) &=& \p{\input{figs/postLoS.tex}} \\
    \\
    (\bm\Sigma_{97.5\%} | \bm Y) &=& \p{\input{figs/postHiS.tex}} \\
\end{array}
\]



Words words words.  Words words words.  Words words words.  Words words words.  Words words words.
Words words words.  Words words words.  Words words words.  Words words words.  Words words words.
Words words words.  Words words words.  Words words words.  Words words words.  Words words words.
Words words words.  Words words words.  Words words words.  Words words words.  Words words words.
Words words words.  Words words words.  Words words words.  Words words words.  Words words words.


\begin{figure*}
  \centering
  \includegraphics[scale=.55]{figs/expTheftCounty.pdf}
  \caption{Map of California thefts per 100,000 people for a selected of counties. The left-most plot
  shows the number of robberies per 100,100 for each of the counties. The color code in the shows that
greener areas have fewer robberies (safer) and redder areas have more robberies (more dangerous).}
  \label{fig:eachtheft}
\end{figure*}

\begin{figure*}
  \centering
  \includegraphics[scale=.55]{figs/expAllTheft.pdf}
  \caption{Map of California thefts per 100,000 people for a selected of counties. The left-most plot
  shows the number of robberies per 100,100 for each of the counties. The color code in the shows that
greener areas have fewer robberies (safer) and redder areas have more robberies (more dangerous).}
  \label{fig:alltheft}
\end{figure*}



\begin{figure}
  \centering
  \includegraphics[scale=.5]{figs/sc.pdf}
  \caption{Posterior distribution of the mean parameter.}
  \label{fig:scprob}
\end{figure}

Words words words.  Words words words.  Words words words.  Words words words.  Words words words.




\end{document}

